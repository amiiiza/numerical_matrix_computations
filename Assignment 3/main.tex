\documentclass[12pt]{article}
\usepackage[paper=letterpaper,margin=2cm]{geometry}
\usepackage{amsmath}
\usepackage{amssymb}
\usepackage{amsfonts}
\usepackage{tikz}
\usepackage{newtxtext, newtxmath}
\usepackage{enumitem}
\usepackage{titling}
\usepackage[colorlinks=true]{hyperref}

\setlength{\droptitle}{-6em}

% Enter the specific assignment number and topic of that assignment below, and replace "Your Name" with your actual name.
\title{MS-E1651 - Numerical Matrix Computations
\\Exercise 3}
\author{Amirreza Akbari}
\date{\today}

\begin{document}
\maketitle
\begin{enumerate}[leftmargin=\labelsep]
	\item (P40)\\[0.5em]
\textbf{Solution:} 
	
(a) Given:
$
(A + uv^T)\hat{x} = b \quad \text{(1)}
$
$
Ax = b \quad \text{(2)}
$

We want to show that:

$
(A + uv^T)w = uv^Tx \quad \text{(3)}
$

Now, let's evaluate $(A + uv^T)w$ on the left side of equation (3) using equation (4):


\begin{align*}
(A + uv^T)w &= (A + uv^T)x - (A + uv^T)\hat{x}\\
&= Ax + uv^Tx - (A + uv^T)\hat{x} \\
&= b + uv^Tx - b  \\
&= uv^Tx
\end{align*}

(b) We know that:
\begin{align*}
(A + uv^T)w &= uv^Tx\\
Aw + uv^Tw & = uv^Tx\\
A^{-1}Aw + A^{-1}uv^Tw &= A^{-1}uv^Tx\\
\rightarrow w &= A^{-1}uv^Tx - A^{-1}uv^Tw\\
w &= A^{-1}u(v^Tx) - A^{-1}u(v^Tw)
\end{align*}
Both $(v^Tw)$ and $(v^Tx)$ are in $\mathbb{R}$ and they are scalar. Thus we can have above equation as follows:
$$w = (v^Tw - v^Tx)A^{-1}u$$
and if we put $\alpha = (v^Tw - v^Tx)$ we would have the result that $w = \alpha A^{-1}u$ for some $\alpha \in \mathbb{R}$.

(c) Now, let's find the value of $\alpha$ using these results.

Starting from the equation in \textbf{(a)}: $$(A + uv^T)w = uv^Tx$$.

Substituting $w = \alpha A^{-1}u$ from \textbf{(b)}:

$$(A + uv^T)(\alpha A^{-1}u) = uv^Tx$$

Now, let's simplify this equation:

$$\alpha(A + uv^T)A^{-1}u = uv^TA^{-1}b$$

Both terms $\alpha(A + uv^T)A^{-1}$ and $v^TA^{-1}b$are scalar and both are coeficient of $u$ in both side of equation, so they must be equal. Thus we have

$$\alpha(A + uv^T)A^{-1} = v^TA^{-1}b $$

$$\alpha = \frac{v^TA^{-1}b}{(A + uv^T)A^{-1}} $$

$$\alpha = \frac{v^TA^{-1}b}{(1 + uv^TA^{-1})} $$

Now, let's find $\hat{x}$ using the result from \textbf{(b)}:

$$w = \alpha A^{-1}u$$

$$w = \frac{v^TA^{-1}b}{1 + v^TA^{-1}u}A^{-1}u$$

Now, let's express $\hat{x}$ in terms of $w$:

$$\hat{x} = A^{-1}b - w$$

$$\hat{x} = A^{-1}b - \frac{v^TA^{-1}b}{1 + v^TA^{-1}u}A^{-1}u$$

and since $v^TA^{-1}b$ is a scalar we can write as follows: 

$$\hat{x} = A^{-1}b - \frac{A^{-1}uv^TA^{-1}b}{1 + v^TA^{-1}u}$$
\item (P41)
\textbf{Solution: }

(a) Based on the definition we have

$$(U\Sigma V^T - U\delta \Sigma V^T)e = b$$

Multiplying both side of equation by $U^{T}$ from gives us

$$(\Sigma V^T - \delta \Sigma V^T)e = U^{T}b$$
$$(\Sigma - \delta \Sigma)V^Te = U^{T}b$$

Then we would have $(\Sigma - \delta \Sigma) = diag(\sigma_1 - \delta \sigma_1, \cdots, \sigma_n - \delta \sigma_n)$
and if we define matrix $\hat{\Sigma} = diag(\frac{\sigma_1^{-1}}{1 - \sigma_1^{-1}\delta \sigma_1}, \cdots, \frac{\sigma_n^{-1}}{1 - \sigma_n^{-1}\delta \sigma_n})$, we will have $$\hat{\Sigma}(\Sigma - \delta \Sigma) = (\Sigma - \delta \Sigma)\hat{\Sigma} = I$$

So if multiply both side of equation by $\hat{\Sigma}$, we will have

$$V^Te = \hat{\Sigma}U^{T}b$$

Moreover it is enough to multiply by $V$ from right in both side of equation, so we will have 

$$e = V\hat{\Sigma}U^{T}b$$

(b) First of all, we know that inverse of $A$ is 
$$A^{-1} = V\Sigma^{-1}U^{T}$$
In this problem we are going to use unitarily invariant property, which is

$$||UA||_2 = ||A||_2$$

if $U$ is unitary matrix. Since norm-2 of $b$ is $1$, we would have

\begin{align}
	e &\leq \frac{||A^{-1}||_2}{1 - ||A^{-1}||_2\,||\delta A||_2}\\e &\leq \frac{||V\Sigma^{-1}U^{T}||_2}{1 - ||V\Sigma^{-1}U^{T}||_2\,||U\delta\hat{\Sigma} V^{T}||_2}
	\\ e &\leq \frac{||\Sigma^{-1}U^{T}||_2}{1 - ||\Sigma^{-1}U^{T}||_2\,||\delta\hat{\Sigma} V^{T}||_2}\\e &\leq \frac{||\Sigma^{-1}||_2}{1 - ||\Sigma^{-1}||_2\,||\delta\hat{\Sigma}||_2}
\end{align}
So we would have
$$e \leq \frac{\max_{i} |\sigma_i|}{1 - (\max_{i} |\frac{1}{\sigma_i}|)(\max_i |\frac{\sigma_i^{-1}}{1 - \sigma_i^{-1}\delta \sigma_i}|)}$$
\item (P43)
\textbf{Solution: }

(a) 

$$(345)_{10} = (101011001)_2 \approx (1.010)_{2} \times 2^{8}$$

$$(\frac{1}{3})_{10} = (0.0101010101....)_2 \approx (1.011)_{2} \times 2^{-2}$$

(b)

$$(1.010)_{2} \times 2^{8} = 320 \rightarrow |\hat{x_1} - x_1| = 25$$

$$(1.011)_{2} \times 2^{-2} = \frac{11}{32} \rightarrow |\hat{x_2} - x_2| = \frac{1}{96}$$

(c)

$$ u = \frac{1}{2^{3}} = \frac{1}{8}$$

(d)

\begin{align}
\hat{x_1} + \hat{x_2} = (10100000000.00)_{2} \times 2^{-2} + (1.011)_{2} \times 2^{-2} &= (10100000001.011)_{2} \times 2^{-2}\\
& = (1.010)_{2} \times 2^{8}
\end{align}

\item (P47) 
\textbf{Solution: } 

(a) In this part, we use the previous problem P46. Then, for all $i = 2,\cdots,\,n$, we write $1+u$ instead of $1+\delta_i$, and for all $k = 1,\cdots,n$, we write $1 + u$ instead of $1 + \hat{\delta}$. In the other word, we are trying to make $|fl(\Sigma x_i y_i) - \Sigma x_i y_i|$ greater, and also for each $i$, we consider the inner multiplication like $\Sigma_{k=1}^{n} (1+\delta_i)$.

\begin{align}
	|fl(\Sigma x_i y_i) - \Sigma x_i y_i| \leq |\Sigma x_i y_i (1+u)^n - x_iy_i| 
\end{align}
Based on Lemma (1.5), we have:
\begin{align}
	|fl(\Sigma x_i y_i) - \Sigma x_i y_i| \leq |\Sigma x_i y_i (1+u)^n - x_iy_i| \leq |\Sigma(1 + \frac{nu}{1-nu}) x_i y_i - \Sigma x_i y_i| = \frac{nu}{1-nu} |x^T y|\\ 
	\rightarrow \frac{nu}{1-nu} |x^T y| \leq \frac{nu}{1-nu} ||x||\,||y|| \leq \frac{nu}{1-nu}
\end{align}
In the last part, we used Cauchy–Schwarz inequality to conclude the inequality.

(b) Note that since $U$ and $V$ are unitary matrix, each row and column of them has lenght exactly $1$. Thus, if we name row of $U$, $R_i^T$ and column of $V$, $C_j$, for all $i$ and $j$, $$||R_i||_2=||C_j||_2 = 1$$
So conditions of part (a) hold here, and we can use part (a). Also note that $(UV)_{ij} = R_i^T C_j$. So $|E_{ij}| = |fl(R_i^TC_j)-(UV)_{ij}| = |fl(R_i^TC_j)-R_i^T C_j|$. Moreover, based on part (a), we know that $|fl(R_i^TC_j)-R_i^T C_j| \leq \frac{nu}{1 - nu}$.
So we can write 
$$fl(UV) = UV - E$$
\end{enumerate}
\end{document}
