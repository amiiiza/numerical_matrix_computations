\documentclass[12pt]{article}
\usepackage[paper=letterpaper,margin=2cm]{geometry}
\usepackage{amsmath}
\usepackage{amssymb}
\usepackage{amsfonts}
\usepackage{tikz}
\usepackage{newtxtext, newtxmath}
\usepackage{enumitem}
\usepackage{titling}
\usepackage[colorlinks=true]{hyperref}

\setlength{\droptitle}{-6em}

% Enter the specific assignment number and topic of that assignment below, and replace "Your Name" with your actual name.
\title{MS-E1651 - Numerical Matrix Computations
\\Exercise 2}
\author{Amirreza Akbari}
\date{\today}

\begin{document}
\maketitle
\begin{enumerate}[leftmargin=\labelsep]
	\item (P29) 

\textbf{Solution:} 
	
(a) We are given \(A = LL^T\). We want to prove that \(A\) is symmetric, i.e., \(A = A^T\).

Let's compute the transpose of \(A^T\):

\[
(A^T)^T = A
\]

Now, let's compute the transpose of \(LL^T\):

\[
(LL^T)^T = (L^T)^T L^T = LL^T
\]

Since \((LL^T)^T\) is equal to \(LL^T\) and \((A^T)^T\) is equal to \(A\), we can conclude that \(A\) is symmetric. Using the definition of (1.27), show that if a matrix \(A\) has a decomposition \(A = LL^T\), it must be positive definite.

To show this, let's use the definition of positive definiteness:

A matrix \(A\) is positive definite if for any nonzero vector \(x\), the expression \(x^TAx\) is greater than zero. Given \(A = LL^T\), we want to show that \(x^TAx > 0\) for all nonzero \(x\). Let \(x\) be a nonzero vector, then:

\[
x^TAx = x^TLL^Tx = (L^Tx)^T(L^Tx) = \|L^Tx\|^2 \geq 0
\]

Since the squared norm of any vector is non-negative (\(\|v\|^2 \geq 0\) for any vector \(v\)), we have shown that \(x^TAx \geq 0\) for all nonzero \(x\).

Therefore, matrix \(A\) is positive definite.

(b) To find the Cholesky decomposition of a matrix \(A\) using the recursive definition:

For \(n = 1\), \(rchol(A) = \sqrt{A}\).

For \(n > 1\), we split \(A\) as follows:
\[
A = \begin{bmatrix}
a_{11} & \mathbf{a}_{21}^T \\
\mathbf{a}_{21} & A_{22}
\end{bmatrix}
\]

Let \(L_2 = rchol\left(A_{22} - \frac{\mathbf{a}_{21}\mathbf{a}_{21}^T}{a_{11}}\right)\).

By equation (1.31), we have:
\[
rchol(A) = \begin{bmatrix}
\sqrt{a_{11}} & \mathbf{0} \\
\frac{1}{\sqrt{a_{11}}} \mathbf{a}_{21} & L_2
\end{bmatrix}
\] 

Compute the Cholesky decomposition of the matrix:
\[
A = \begin{bmatrix}
1 & 2 & 2 \\
2 & 8 & 4 \\
2 & 4 & 15
\end{bmatrix}
\]

We start by splitting \(A\) into the specified form:
\[
A = \begin{bmatrix}
1 & \begin{bmatrix} 2 \\ 2 \end{bmatrix}^T \\
\begin{bmatrix} 2 \\ 2 \end{bmatrix} & \begin{bmatrix} 8 & 4 \\ 4 & 15 \end{bmatrix}
\end{bmatrix}
\]

Next, we compute \(L_2\) for the bottom-right submatrix using the recursive definition:
\[
L_2 = rchol\left(\begin{bmatrix} 8 & 4 \\ 4 & 15 \end{bmatrix} - \frac{\begin{bmatrix} 2 \\ 2 \end{bmatrix} \begin{bmatrix} 2 & 2 \end{bmatrix}}{1}\right)
\]

Now, compute the subtraction and \(L_2\):

\[
L_2 = rchol\left(\begin{bmatrix} 8 & 4 \\ 4 & 15 \end{bmatrix} - \begin{bmatrix} 4 & 4 \\ 4 & 4 \end{bmatrix}\right)
\]

so we have 

\[
L_2 = rchol\left(\begin{bmatrix} 4 & 0 \\ 0 & 11 \end{bmatrix}\right)
\]

Now, recursively compute the Cholesky decomposition of this \(2 \times 2\) matrix. We start by splitting $\left(\begin{bmatrix} 4 & 0 \\ 0 & 11 \end{bmatrix}\right)$ into the specified form:
\[
\begin{bmatrix}
4 & \begin{bmatrix} 0 \end{bmatrix}^T \\ 
\begin{bmatrix} 0 \end{bmatrix} & \begin{bmatrix} 11 \end{bmatrix}
\end{bmatrix}
\]

so we have:

\[
L_2 = \begin{bmatrix}
2 &  0  \\ 
 0  & \sqrt{11} 
\end{bmatrix}
\]

So, the final result for \(rchol(A)\) can be constructed using the recursive formula:

\[
rchol(A) = \begin{bmatrix} 1 & 0 & 0 \\ 2 & 2 & 0 \\ 2 & 0 & \sqrt{11} \end{bmatrix}
\]

(c) 
Let $F$ be the matrix:
\[
F = \begin{bmatrix}
0 & 1 & 0 \\
0 & 0 & 1 \\
1 & 0 & 0
\end{bmatrix}
\]

And let $A$ be the matrix:
\[
A = \begin{bmatrix}
1 & 2 & 2 \\
2 & 8 & 4 \\
2 & 4 & 15
\end{bmatrix}
\]

We want to compute $F^TAF$:

1. Calculate $F^T$, which is the transpose of matrix $F$:
\[
F^T = \begin{bmatrix}
0 & 0 & 1 \\
1 & 0 & 0 \\
0 & 1 & 0
\end{bmatrix}
\]

2. Now, compute $F^TAF$ by performing the matrix multiplications:
\[
F^TAF = \begin{bmatrix}
0 & 0 & 1 \\
1 & 0 & 0 \\
0 & 1 & 0
\end{bmatrix}
\begin{bmatrix}
1 & 2 & 2 \\
2 & 8 & 4 \\
2 & 4 & 15
\end{bmatrix}
\begin{bmatrix}
0 & 0 & 1 \\
1 & 0 & 0 \\
0 & 1 & 0
\end{bmatrix}
\]

3. Calculate the product $F^TAF$ by performing the multiplications:
\[
F^TAF = \begin{bmatrix}
0 & 0 & 1 \\
1 & 0 & 0 \\
0 & 1 & 0
\end{bmatrix}
\begin{bmatrix}
1 & 2 & 2 \\
2 & 8 & 4 \\
2 & 4 & 15
\end{bmatrix}
\begin{bmatrix}
0 & 0 & 1 \\
1 & 0 & 0 \\
0 & 1 & 0
\end{bmatrix}
=
\begin{bmatrix}
15 & 2 & 4 \\
2 & 1 & 2 \\
4 & 2 & 8
\end{bmatrix}
\]

So, with the updated matrix $F$, the result of $F^TAF$ is the same as the given matrix $A$:
\[
F^TAF =
\begin{bmatrix}
15 & 2 & 4 \\
2 & 1 & 2 \\
4 & 2 & 8
\end{bmatrix}
\]

The matrix 
$F$ has a trivial null space because it is a permutation matrix. On the other hand, we found Cholesky decomposition for matrix $A$ and based on the part (a), matrix $A$ is positive definite. According to Lemma 1.3, $F^TAF$ is also positive definite.

\newpage
\item (P30) 

\textbf{Solution:} 

(a) The graph $G(A)$ can be visualized as follows:

\begin{center}
\begin{tikzpicture}
  % Vertices
  \node[circle, draw] (1) at (0, 0) {1};
  \node[circle, draw] (2) at (2, 0) {2};
  \node[circle, draw] (3) at (4, 0) {3};
  \node[circle, draw] (4) at (6, 0) {4};
  \node[circle, draw] (5) at (2, -1.5) {5};
  % Edges
  \draw (1) -- (2);
  \draw (2) -- (3);
  \draw (2) -- (5);
  \draw (3) -- (4);
\end{tikzpicture}
\end{center}

So, this is the graph $G(A)$ corresponding to the matrix $A$. It consists of five vertices and the edges connecting them as described above.

(b) For each vertex $i \in V(A)$, we will compute the set $\text{reach}(i, \{1, \ldots, i - 1\})$, which represents the set of vertices that can be reached from vertex $i$ by following set along the edges of the graph $G(A)$. Let's compute these sets:

1. For vertex 1:
   $\text{reach}(1, \emptyset) = \{2\}$ 

2. For vertex 2:
   $\text{reach}(2, \{1\}) = \{3, 5\}$ 

3. For vertex 3:
   $\text{reach}(3, \{1, 2\}) = \{4, 5\}$ 

4. For vertex 4:
   $\text{reach}(4, \{1, 2, 3\}) = \{5\}$ 

5. For vertex 5:
   $\text{reach}(5, \{1, 2, 3, 4\}) = \emptyset$ 

(c) 

\begin{itemize}
\item off-diagonal non-zeros on column 1 are $\text{reach}(1, \emptyset) = \{2\}$.
\item off-diagonal non-zeros on column 2 are $\text{reach}(2, \{1\}) = \{3, 5\}$.
\item off-diagonal non-zeros on column 3 are $\text{reach}(3, \{1, 2\}) = \{4, 5\}$.
\item off-diagonal non-zeros on column 4 are $\text{reach}(4, \{1, 2, 3\}) = \{5\}$
\item off-diagonal non-zeros on column 5 are $\text{reach}(5, \{1, 2, 3, 4\}) = \emptyset$.

\end{itemize}

The non-zeros of the computed factor are

\[
\begin{bmatrix}
x & 0 & 0 & 0 & 0 \\
x & x & 0 & 0 & 0 \\
0 & x & x & 0 & 0 \\
0 & 0 & x & x & 0 \\
0 & x & x & x & x \\
\end{bmatrix}
\]

(d) The code is in p35d.m
\end{enumerate}
\end{document}
